\documentclass{article} % For LaTeX2e
\usepackage{cos424,times}
\usepackage{hyperref}
\usepackage{url}
\usepackage{graphicx}
\usepackage{amsmath}
\usepackage{multirow}
\usepackage{mathtools}


\bibliographystyle{plos2009}


\title{meth}


\author{
Chaney C. Lin\\
Department of Physics\\
Princeton University\\
\texttt{chaneyl@princeton.edu} \\
\And
Liangsheng Zhang\\
Department of Physics\\
Princeton University\\
\texttt{liangshe@princeton.edu} \\
}

\newcommand{\fix}{\marginpar{FIX}}
\newcommand{\new}{\marginpar{NEW}}
\DeclarePairedDelimiter{\norm}{\lVert}{\rVert}


\begin{document}

\maketitle

\begin{abstract}
\end{abstract}

todo:
\begin{itemize}
\item update so that $d$ refers to position
\item mention that there are other variables present, but that we only look at position (since all other variables are...)
\item update all fitted $\beta$ to include hats
\end{itemize}
\section{Introduction}

External factors affect how genes are expressed. One such factor is DNA methylation, a biochemical process in which a methyl group (CH$_3$) attaches to a cytosine nucleotide, usually where a cytosine (C) neighbors a guanine (G) (such a neighboring pair of C and G is called a CpG site, or CG site).

Just as our understanding of genetics develops with better statistics and more data, so it is with DNA methylation. However, assaying methylation levels across the full genome remains prohibitively expensive; more affordable partial assays are available, but for these, the difficulty is predicting the methylation levels (MLs) of unmeasured sites, so-called imputing. This is what we aim to achieve in our project: a reliable model for imputing.

\section{Data Description}

The training data is a set of $N = 32$ whole genome bisulfite sequences, from Ziller et al ~\cite{ziller2013charting}. These are the reference sequences. We refer to them as \texttt{training}. The test data is a partial assay, also from Ziller et al ~\cite{ziller2013charting}. Two separate sets are included in the test data: one called \texttt{sample}, another called \texttt{test}; both refer to the exact same sequence, but the values are mostly unobserved in \texttt{sample}, while observed in \texttt{test}. Data for twenty chromosomes were provided; in this report, we work only with the first chromosome.

For each sequence $i$, we have the MLs $\left\{\beta_{i,d}\right\}$ at CpG sites, where $d$ is the starting position of a given site, and where $\beta \in [0,1]$ is the fraction of methylated reads at the site from this sequence $i$ with starting position $d$. Starting and ending positions for each site were provided, but the difference between each pair is the same; so henceforth, by position, we refer only to the start position. The end position is redundant. The positions are the same for all sequences in \texttt{training} and \texttt{sample}. We denote the \texttt{sample} MLs as $\beta_{0,d}$, and the \texttt{training} MLs as $\beta_{i,d}$ where $1 \leq i \leq N$. Each sequence contains 379,551 sites.

Other features are provided, e.g. strand. We do not fit to these features. We will mention in this report whenever they are implicitly incorporated in $d$.

Models are fit to \texttt{training}, to predict the \texttt{sample} values $\hat{\beta}_{0,d}$, which are then compared against the observed \texttt{test} values $\beta_{0,d}$ to estimate the models' prediction error, according to Eqn. (\ref{metric}).



\section{Methods}

\subsection{Data processing}

The data set was downloaded on March 20, 2015 from Bianca Dumitrescu's COS424 directory on Princeton's Nobel cluster. Both \texttt{training} and \texttt{sample} contain sites with unmeasured MLs, labeled ``NaN".

The only preprocessing is imputing on \texttt{training} to replace the NaN values. We have chosen to set a NaN value as the average of observed values with the same $d$, in the other reference sequences. 

A subset of the \texttt{sample} MLs has been observed. We denote this subset by $\Omega$.

\subsection{Models}

The models were applied using built-in functions of the scikit-learn Python library \cite{scikit-learn}. Default parametrizations were used, unless otherwise specified. Below, we define each model, explain our motivation for them, and summarize their performance.

\begin{enumerate}
\item \emph{Naive mean model (NMM)}. The first model we implemented was a naive mean model. It performed reasonably well, according to Eqn. (\ref{metric}), and was simple and fast. It initially served as the benchmark for all of our subsequent models. This model predicts using the mean of the reference sequences
\[ \hat{\beta}_{0,d} = \underset{i}{\text{avg}} \left\{ \beta_{i,d}\right\}\]
Such a model would be optimal if for each $d$, the only information available were $\beta_{i,d}$.

\item \emph{Simple linear regression (SLR1)}. The next model was essentially a simple linear regression. For each site $d \notin \Omega$, a simple linear regression is performed, using features $\left\{ \beta_{i,x} \right\}_{x \in \Omega}$ to fit the value $\beta_{i,d}$. Because $i$ runs from $1$ to $N$, there are $N$ data points (for each $d \notin \Omega)$. The resulting parameters $\left\{ a_x\right\}_{x \in \Omega}$ are used to predict $\beta_{0,d}$ by
\begin{equation} \label{beta.sites}\hat{\beta}_{0,d} = \sum_{x \in \Omega} a_x \beta_{0,x} \ .\end{equation}
In general, for different $d'$, the parameters $\left\{a_x'\right\}$ will differ from $\left\{a_x\right\}$. The parameters $\left\{a_x\right\}$ for a given $d$ thus implicitly incorporates all of the information given by $d$.

On one hand, SLR1 performed significantly better than NMM, according to Eqn. (\ref{metric}), but on the other hand, it was significantly slower, given the number of linear regressions that need to be done. Another issue is that the predictions need not lie in the range $[0,1]$.

\item \emph{Generalized linear model (GLM)}. To solve the issue of the prediction range, we then implemented a generalized linear model, using the logit link function $f(\beta): [0,1] \to (-\infty,\infty)$, which is of the form
\[ f(\beta) = \ln(\beta^{-1} - 1) \ .\]

The computational time is roughly the same as the linear regression model, but performs slightly worse.

\item \emph{Regularized linear regression (RegLR)}. We then sought to reduce the dimensionality of the feature space, by introducing regularizers in the linear regression. We looked at both LASSO regression and ridge regression. The regularization constant $\alpha$ was determined through cross validation.

Ridge regression was three times faster than simple linear regression; LASSO was three times slower---in fact, LASSO was the slowest of all models we considered in our study. The performance of both was comparable to simple linear regression.

\item \emph{$K$-means clustering ($K$means)}. To significantly reduce the computational time, we then implemented a model based on $K$-means clustering. This model can be considered as an approximation to our simple linear regression.

For $d \in \Omega$, $B_{d,k}$ refers to a vector $ \left(\beta_{1,d},\dots,\beta_{N,d}\right)$ that has been classified into the $k$th cluster using $K$-means clustering. We assign the same classification to the corresponding site in $\texttt{sample}$, which we denote by $\beta_{0,d,k}$. A simple linear regression is performed, using as features the centroids $C_k = \text{avg}\left\{B_{d,z} : d \in \Omega, z = k\right\}$ of the $K$ clusters to predict $B_{d}$ for $d \notin \Omega$. The resulting parameters $a_k$ are then used to predict $\beta_{0,d}$ by
\[ \hat{\beta}_{0,d} = \sum_{1 \leq k \leq K} a_k C_{0,k}\]
where $C_{0,k} = \text{avg}\left\{\beta_{0,d,z} : d \in \Omega, z = k\right\}$ are the centroids of the $K$ clusters in \texttt{sample}.

This model significantly reduces the dimensionality of the feature space when $K$ is small. We looked at $K = 2^n$ for $3 \leq n \leq 9$. Each was significantly faster than the other regression models, and with only one exception, they performed relatively well.

\item \emph{Simple linear regression 2 (SLR2)}. We then performed another simple linear regression model, looking at the data from a different perspective, using as features $\left\{ \beta_{i,d} \right\}_{1\leq i \leq N}$ to fit the value $\beta_{0,d}$. Because every position $d \in \Omega$ provides one data point, the total number of data points is the size of $\Omega$. The resulting parameters $\left\{ a_{i}\right\}_{1\leq i \leq N}$ are used to predict $\beta_{0,d}$ by
\begin{equation} \label{beta.samples} \hat{\beta}_{0,d} = \sum_{1 \leq i \leq N} a_i \beta_{i,d} \ .\end{equation}

This model was fast, and was the best performing amongst the models we looked at.

\end{enumerate}

\subsection{Evaluation}

The metric used to evaluate the predictions $\left\{ \hat{\beta}_{0,d}\right\}$ for a given model is 
\begin{equation} \label{metric}
\epsilon^2 = \frac{ 1}{|\Gamma| S^{2}} \sum_{d \in \Gamma} \left( \hat{\beta}_{0,d} -  \beta_{0,d}\right)^2 
\end{equation}
where $\Gamma$ is the set of sites with observed values (i.e. not NaN) in \texttt{test} that are not observed in \texttt{sample}. The $\left\{\beta_{0,d}\right\}$ are the known values in \texttt{test}. $S^2$ is the variance of $\left\{\beta_{0,d}\right\}_{d \in \Gamma}$. This metric is non-negative, and smaller values correspond to better predictions.

If $\hat{\beta}$ were fitted by a linear regression on $\beta$, then $\epsilon^2 = 1 - R^2$, where $R^2$ is the coefficient of determination. Alternatively, it can be interpreted as how good the prediction is, relative to just predicting with one number, the sample mean; this can be seen from the definition of $S^2$, and observing that without the $S^2$, the right hand side is simply the average error squared.

\section{Results}

\begin{table}[htbp]
\small
   \centering
   \begin{tabular}{@{}|c|c|c|c|@{}} % Column formatting, @{} suppresses leading/trailing space 
  \cline{1-4}
Model & $t_F$ & $t_C$ & $\epsilon^2$ \\ \hline
\hline
SLR2 & 00:00:23 & - & 0.12617176 \\ \hline
RidgeLR & 01:47:21 & - & 0.12680164 \\ \hline
SLR1 & 07:00:54 & - & 0.12684927 \\ \hline
GLM & 07:00:50 & - & 0.12804621 \\ \hline
256means & 00:12:28 & 00:02:21 & 0.13797524 \\ \hline
512means & 00:16:53 & 00:04:08 & 0.14206517 \\ \hline
128means & 00:09:58 & 00:01:06 & 0.14661424 \\ \hline
64means & 00:09:09 & 00:00:31 & 0.15221729 \\ \hline
LassoLR & 21:35:56 & - & 0.15377764 \\ \hline
16means & 00:05:22 & 00:00:20 & 0.17631774 \\ \hline
8means & 00:04:36 & 00:00:14 & 0.18287992 \\ \hline
NMM & 00:00:08 & - & 0.50788960 \\ \hline
32means & 00:07:44 & 00:00:26 & 0.81595437 \\ \hline
   \end{tabular}
   \caption{For each model, we report: (i) the fitting and prediction time $t_F$, in the format hh:mm:ss; (ii) the clustering time $t_C$, for the $K$-means models, in the format hh:mm:ss; and (iii) the prediction error $\epsilon^2$, defined in (\ref{metric}). Models are sorted (in decreasing order) by $\epsilon^2$.}
   \label{tab:models}
\end{table}

\subsection{Computational speed}

The fastest model to predict was NMM, given that it does not have to do any fitting. Next fastest was SLR2, which performed only one fitting. All other models performed multiple iterations of linear regression, so were significantly slower. Of these, the $K$-means models were the fastest, as they reduce the dimensionality of the feature space in the clustering step. The remaining regression models were much slower than $K$-means, with Lasso regularized regression performing the slowest of all. Among the $K$-means models, one sees that the clustering time $t_C$ and the fitting and predicting time $t_F$ increases with the number of clusters, as expected: more clusters means more centroids and more features.

Interpretation of error

Difference in speed

Interpretation of results

Comparing two data sets

\subsection{Generalization error}



(questions in problem writeup)
\begin{itemize}
\item what features were most important for prediction? what do features tell us about problem
\item were some reference samples more predictive of the held out sample than others?
\item what did distribution of prediction errors look like? approximately gaussian and zero centered, or heavy tail, or what?
\end{itemize}


include specific examples of CpG sites or features that highlight behavior of the models

provide analysis code


\section{Discussion}

\subsection{Assumptions of our models}
\begin{enumerate}
\item [Class 1.] For a given site $d$ 
\item [Class 2.]
\end{enumerate}

\section{Conclusion}
work with more chromosomes


\bibliography{ref}

\end{document}
